\newpage
\chapter{Theoretical Basis}
In the first section, this chapter describes the concept of URL and how resources are searched and filtered using URL parameters. The second section describes the concept of a browser extension. Furthermore, the choice of implementation browser is defined with which it is possible to develop an extension. The third chapter covers the architecture of a Chrome Extension. Finally, the framework for the extension's UI implementation is elucidated.

% Uniform Resource Locator
\section{Uniform Resource Locator}
Uniform Resource Locator, or URL, is a compact string representation for a resource available via the Internet \autocite{berners1994uniform}. URLs are used to "locate" resources, by providing an abstract identification of the resource location. These resources could be an image, a CSS file, an HTML page, etc. In practice, there are a few exceptions, the most frequent of which is a URL leading to a resource that has either relocated or vanished. After locating a resource, a system may carry out a number of actions on it, which can be described by phrases like "access", "update", "replace", and "find attributes". For each URL scheme, only the "access" method needs to be supplied. Here is an example of an HTTP URL: \texttt{http://www.example.com/software/index.html}

\subsection{Anatomy of a URL}
A URL is made up of various components, some of which are required and others which are not \autocite{mozilla2022url}. The most important parts are provided in the following sections:

\subsection*{Scheme}
The scheme, which indicates the protocol that the browser must use to request the resource, is the first part of the URL. A protocol is a set method for exchanging or transferring data around a computer network. The most common protocol is HTTP which stands for Hypertext Transfer Protocol. Nowadays most websites use HTTPS protocol which stands for Hypertext Transfer Protocol Secure.

\subsection*{Authority}
The authority is then separated from the scheme by the character pattern \texttt{://}. If the authority is present, it includes both the host (e.g., \texttt{www.example.com}) and the port (80), separated by a colon:

\begin{itemize}
  \item The host name identifies the host that holds the resource. A server provides services in the name of the host, but hosts and servers do not have a one-to-one mapping.
  \item The port number denotes the technical "gateway" used to access the web server's resources. It is typically omitted if the web server grants access to its resources via the HTTP protocol's standard ports. Otherwise, it is required.
\end{itemize}

\subsection*{Path}
The path identifies the specific resource in the host that the web client wants to access. For example, \texttt{/software/htp/cics/index.html}.

\subsection*{Query String}
A query is commonly found in the URL of dynamic pages. and is represented by a question mark followed by one or more parameters. The query directly follows the host name, path or port number. For example, this URL was generated by Google when doing a search for the word "query":

\begin{center}
  \url{https://www.google.com/search?q=query&rlz=1C5GCEM_enDE993DE993&oq=query&aqs=chrome..69i57j0i512l4j69i60l3.938j0j7&sourceid=chrome&ie=UTF-8}
\end{center}

\noindent This is the query part:

\begin{center}
  \url{?q=query&rlz=1C5GCEM_enDE993DE993&oq=query&aqs=chrome..69i57j0i512l4j69i60l3.938j0j7&sourceid=chrome&ie=UTF-8}
\end{center}

\subsection*{Anchor}
An anchor is a type of "bookmark" within the resource that instructs the browser to display the content located at that "bookmarked" location. For example, in an HTML document, the browser will scroll to the point where the anchor is defined; in a video or audio document, the browser will attempt to navigate to the time the anchor represents. It is important to note that the part following the \texttt{\#}, also known as the fragment identifier, is never sent to the server with the request.

\subsection{Search and Filter URL Parameters}
Search and filter URL parameters are parameters or query strings that add information to a specific URL. A search or filter parameter facilitates the search for a specific phrase or keyword within search engine results. They include what is requested while excluding irrelevant content. Aside from the functions mentioned above, the most common use cases for parameters are tracking, pagination, site search, sorting and filtering.

%Site Search
\section{Site Search}
Providing a search function that searches your Web pages is a design strategy that offers users a way to find content \autocite{w3c2016search}. Users can find content by searching for specific words or phrases without having to understand or navigate the site's structure. This can be a faster or easier way to find content on large sites. A great site search function is specific to the website and not only constantly indexes the site to ensure the most recent content is easily accessible, but it also guides users as they explore a website's content, assisting them in discovering content they may not have known they were interested in. The best site search products delight users by allowing them to quickly connect with the content they require while also collecting valuable data about the content and products that visitors are most interested in.


%Filters
\section{Filters}
The process of narrowing down a search based on predefined categories is known as filtering. These categories are frequently broad and based on a single dimension of the product. This allows user to quickly narrow down a large number of products to a more manageable set for further investigation.

Filters are broad categories defined by the business that do not change between searches, and they are frequently used behind the scenes. For example, an online clothing store might use "clothing type" as a filter, with four possible categories: shirts, pants, shoes, and accessories. When a website visitor clicks on "shirts" in the top navigation, the clothing type filter is applied, and the visitor sees only shirts on the results page.


%Facets and faceted search
\section{Facets and faceted search}
Facets, also known as facet filters, enable users to filter results by selecting values along different dimensions or facets \autocite{qu2021study}. It is widely used in e-commerce search engines and digital libraries where documents have rich metadata. Faceted search is a more granular method of finding products and results in a specific, targeted way that broad, one-size-fits-all filters do not allow.

Facets and faceted search are features of a well-designed user interface. Contextual facets that change depending on the item or category drive a user-friendly experience by guiding the user down the quickest path to the best result.
