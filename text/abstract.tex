\newpage
\thispagestyle{empty}
% vertikaler Leerraum
\vspace*{2.2cm}
\noindent %kein Einzug
{\Huge Abstract}\\
\vspace*{1.6cm} \\

% Kopfzeilen (automatisch erzeugt)
\pagestyle{headings}
Filtering resources should be one of the primary focus for users when searching for a product or an article online, in order to reduce the number of options available to them and find the desired item in seconds. This thesis examines the possibility to create a personalized filter suggestions tool using a google chrome extension. These suggestions are based on the user's most entered URL. The more often user visits a web page with its filter within the URL parts, the more likely the filter is suggested.

To properly identify the requirements and features needed for the development of the extension, a requirement analysis process is conducted. The extension allows users to select their most used filters within certain websites without numerous steps. In addition, it enables users to navigate through the list of the visited URLs inside the extension in a manner similar to file directory system. While traversing through the list of the path names, the extension assembles a new URL which the user can then visit.

\newpage
\thispagestyle{empty}
% vertikaler Leerraum
\vspace*{2.2cm}
\noindent %kein Einzug
{\Huge Zusammenfassung}\\
\vspace*{1.6cm} \\

% Kopfzeilen (automatisch erzeugt)
\pagestyle{headings}
Bei der Suche nach einem Produkt oder einem Artikel im Internet sollte das Filtern von Ressourcen im Vordergrund stehen, um die Anzahl der verf\"ugbaren Optionen zu reduzieren und den gew\"unschten Artikel in Sekundenschnelle zu finden. In dieser Arbeit wird die M\"oglichkeit untersucht, mit Hilfe einer Google Chrome Extension ein Tool f\"ur personalisierte Filtervorschl\"age zu erstellen. Diese Vorschl\"age basieren auf der vom Benutzer am h\"aufigsten eingegebenen URL. Je \"ofter der Benutzer eine Webseite mit einem Filter in der URL besucht, desto wahrscheinlicher wird der Filter vorgeschlagen.

Um die Anforderungen und Funktionen, die f\"ur die Entwicklung der Extension ben\"otigt werden, richtig zu identifizieren, wird ein Anforderungsanalyseprozess durchgef\"uhrt. Die Extension erm\"oglicht es den Benutzern, ihre am h\"aufigsten verwendeten Filter auf bestimmten Websites ohne zahlreiche Schritte auszuw\"ahlen. Au{\ss}erdem k\"onnen die Benutzer durch die Liste der besuchten URLs innerhalb der Extension navigieren, \"ahnlich wie in einem Dateiverzeichnis-System. Beim Durchlaufen der Liste mit den Pfadnamen stellt die Extension eine neue URL zusammen, die der Benutzer dann besuchen kann.
