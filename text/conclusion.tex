\newpage
\chapter{Conclusion}
This chapter provides a concise summary of the results and discoveries made in the preceding chapters, as well as an outlook on potential improvements to the system in the near future.

\section{Summary}
A large website with filter options or even a faceted navigation system can be difficult to navigate, especially if the search must be repeated. Toggling on a filter is a simple task, but performing it multiple times can be tedious, especially if the filter is identical. The thesis's primary objective is to enhance the user experience in this particular scenario.

Amazon and Netflix, among other leaders in e-commerce and media, recognize the importance of a good search and navigation experience on their websites. They have invested heavily in making their vast catalogs easily navigable, thereby exposing users to new, valuable products and content. However, there is one aspect on which they place little emphasis: the frequency of visits to an item or product. Amazon and Netflix recommend the most popular global searches to users. These recommendations are not customized to the user and are based on data collected from users around the world. They provide related products, but not the most frequently visited item.

The objective of this thesis is to address the aforementioned problem for websites that support search and filtering. With the extension, not only can users select the most frequently used query parameters for the current host name of the URL, but also return to a previously visited page in a matter of clicks. The extension is exclusive to Chrome users and stores its data in the user's \texttt{chrome.storage} API. Each time a user modifies the URL of a browser tab, the extension saves the new information. More frequently a user visits a page or uses a filter on the page, the higher the URL suggestion will appear in the extension.

\section{Outlook}
The extension proposed in this thesis is far from perfect. Extensions are products that can and should be improved over time. One significant improvement of the extension is the internationalization of the application, which allows users to select their preferred language when using the extension. Due to the fact that the extension's initial focus was on Marta's customer-facing platform, some obstacles were overlooked.

Even though users can exclude URL parameters from the options page, there are still plenty of marketing or UTM parameters\footnote{UTM parameters are five different types of URL parameters that are used by marketers to monitor the success of online marketing campaigns across traffic sources and publishing platforms.} that vary by website. As each website may change these parameters at some point in the future, manually entering these query parameters one at a time is not a long-term solution. It is advisable to conduct a research to identify the marketing parameters that are commonly used by websites and the findings should be listed. The list is then used as the default values of the excluded parameters for the extension.

In addition, some websites use hash values in URL path names, which reduces the readability of the extension. Furthermore, the extension currently works for every website. Another good improvement would be to disable the extension if the website does not allow filtering.
