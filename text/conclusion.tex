\newpage
\chapter{Conclusion}
The results and findings from previous chapters are summarized in this chapter, along with an outlook on future system improvements.

\section{Summary}
It can be difficult to navigate a large website with filter options or even a faceted navigation system, especially if the search must be repeated. Toggling a filter on is a simple task, but doing it several times can be tedious, particularly if the filter is the same. Improving the user experience in this specific scenario is the main goal of the thesis.

Leaders in e-commerce and media, such as Amazon and Netflix, recognize the value of a good search and navigation experience on their websites. They've invested heavily in making their vast catalogs easy to navigate, exposing users to valuable new products and content along the way. However, there is one aspect on which they do not place much emphasis: the number of times an item or product is visited. Amazon and Netflix recommend to users the most popular searches around the world. These recommendations are based on users from all over the world and are not personalized for the user. They provide related products but not the one that the user visits most frequently.

The thesis aims to address the aforementioned issue for websites that support search and filtering. With the extension, users can easily return to a previously visited page with a few clicks. The extension is only available for Chrome users and stores its data on the end user's \texttt{chrome.storage}. Each time a user changes the URL of a tab in their browser, the extension saves the entered information. The more frequently a user visits a page or uses a filter available on the page, the higher the position of the URL suggestion within the extension.

\section{Outlook}
The extension proposed in this thesis is far from perfect. Extensions are products that can and should be improved over time. One significant improvement of the extension is the internationalization of the application, which allows users to select their preferred language when using the extension. Because the extension's initial primary focus is Marta's customer-facing platform, there are some obstacles that were overlooked.

Some websites use hash values in URL path names, which reduces the readability of the extension. Even though users can exclude URL parameters from the options page, there are still plenty of marketing or UTM parameters that vary by website (UTM parameters do not vary). Another useful improvement would be to suggest users directly the top most used filters in the first view, instead of guiding them to traverse through the "URL directories". Furthermore, the extension currently works for every website. Another good improvement would be to disable the extension if the website does not allow filtering.

Last but not least, after adding the internationalization support, the extension could be published to the Chrome Web Store.
