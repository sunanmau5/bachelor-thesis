\newpage
\chapter{Conclusion}
This chapter provides a concise summary of the results and discoveries made in the preceding chapters, as well as an outlook on potential improvements to the system in the near future.

\section{Summary}
A large website with filter options or even a faceted navigation system can be difficult to navigate, especially if the search must be repeated. Toggling on a filter is a simple task, but performing it multiple times can be tedious, especially if the filter is identical. The thesis's primary objective is to enhance the user experience in this particular scenario.

Amazon and Netflix, among other leaders in e-commerce and media, recognize the importance of a good search and navigation experience on their websites. They have invested heavily in making their vast catalogs easily navigable, thereby exposing users to new, valuable products and content. However, there is one aspect on which they place little emphasis: the frequency of visits to an item or product. Amazon and Netflix recommend the most popular global searches to users. These recommendations are not customized to the user and are based on data collected from users around the world. They provide related products, but not the most frequently visited item.

The objective of this thesis is to address the aforementioned problem for websites that support search and filtering. With the extension, users can return to a previously visited page in a matter of clicks. The extension is exclusive to Chrome users and stores its data in the user's \texttt{chrome.storage} directory. Each time a user modifies the URL of a browser tab, the extension saves the new information. More frequently a user visits a page or uses a filter on the page, the higher the URL suggestion will appear in the extension.

\section{Outlook}
The extension proposed in this thesis is far from perfect. Extensions are products that can and should be improved over time. One significant improvement of the extension is the internationalization of the application, which allows users to select their preferred language when using the extension. Due to the fact that the extension's initial focus was on Marta's customer-facing platform, some obstacles were overlooked.

Some websites use hash values in URL path names, which reduces the readability of the extension. Even though users can exclude URL parameters from the options page, there are still plenty of marketing or UTM parameters that vary by website (UTM parameters do not vary). Another useful improvement would be to suggest users directly the top most used filters in the first view, instead of guiding them to traverse through the "URL directories". Furthermore, the extension currently works for every website. Another good improvement would be to disable the extension if the website does not allow filtering.

Last but not least, after adding the internationalization support, the extension could be published to the Chrome Web Store.
