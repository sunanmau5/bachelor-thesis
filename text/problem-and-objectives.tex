\newpage
\section{Problem Statement and Objectives}

It is a common belief in modern society that the more choices, the better--that the human ability to manage, and the human desire for, choice is unlimited. One study showed that the existence of choice increases motivation and enhances performance on doing tasks \cite{zuckerman1978importance}. However, another study has shown that although having more choices might appear desirable, it may sometimes have negative effects on human motivation \cite{iyengar2000choice}. In our digital and website-driven era, these studies can be applied on e-commerce and a solution to this problem is a search-and-filter functionality. Amazon is one of the corporate giants for e-commerce, that implented this feature in their platform.

In our case, marta needs to introduce a continually improved, user-friendly filtering functionality to adjust to the needs of families. An example of a user-friendly filter would be to provide quick filter suggestions which the user can click once and the desired results will be shown, instead of letting users select the same filter manually over and over again. In order to determine which filter suggestions are the most beneficial, frequently used filters need to be identified.
