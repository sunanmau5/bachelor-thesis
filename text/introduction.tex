\newpage
\chapter{Introduction}
The way we access to digital information directly has fundamentally altered how information is retrieved, allowing us to create search engines that can significantly speed up the search process by allowing users to jump straight to the content they are interested in without having to navigate through convoluted systems. This instant gratification far outperforms previous methods of flipping through physical pages. But as digital information access has grown, so has the amount of information that is available on any given subject. As a result, finding the desired product would be difficult even with instant access via search.

The idea that the more options available, the better, and that human desire for choice is limitless are both prevalent in modern society. One study showed that the existence of choice increases motivation and enhances performance on doing tasks \autocite{zuckerman1978importance}. However, another study has shown that although having more choices might appear desirable, it may sometimes have negative effects on human motivation \autocite{iyengar2000choice}. To resolve this issue, filters are introduced. Filters help users find informations faster. Rich information systems have recently begun to provide faceted navigation, which basically extends the idea of filters into a complex structure that attempts to describe all the different aspects of an object in order to maximize flexibility in retrieving information \autocite{whitenton2014filters}. Nevertheless, using this more flexible and more useful tool requires multiple steps, expecially if users repeatedly search for the same information.

\section{Objectives}
The goal of this thesis is to design and implement a filtering suggestions tool, consisting of a client component, to circumvent the numerous steps involved in searching for an article or a product on a website. Thereby, a clear presentation and its development will be discussed as well as the technical background of the extension. The client component is implemented as a Google Chrome extension. To achieve a user-friendly extension, it is critical to provide the user with access to useful and clear information, so the extension to be developed will be compactly customized to one view, allowing the user to see their frequently visited websites.

\section{Structure of the Thesis}
The thesis is divided into four chapters. First, the descriptions of filtering, exploratory search and faceted search are provided. In addition, the anatomy of a URL as well as search and filtering usage in a URL are defined. Finally, the fundamentals of the technologies used are then explained in chapter 2. Further on, in chapter 3 the experimental methodology used for studies conducted with the Chrome extension is described. The extension's implementation and use in a practical situation are discussed in some detail in chapter 4 along with some lessons learned. Subsequently, the extension is evaluated and analyzed. Finally, problems are identified and an outlook for possible improvements of the extension is given.
