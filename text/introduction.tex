\newpage
\section{Introduction}

Marta as a business is currently best described as a marketplace between caregivers and families requiring 24-hour care. 24-hour care can be defined as living in a household with the person in need of care for a certain period of time. This means that caregivers are primarily responsible for basic care and household chores. In addition, they support the person in care\'s relatives in need of assistance in carrying out the activities they wish to do. Marta as a marketplace connecting families with caregivers is competing against more traditional agencies, where it can take several days or even weeks to find a family for a newly signed up caregiver or the other way around.

One way marta can provide a superior experience for both caregivers and families is to speed up the matching process. The caregiver and family inquiry forms are designed to record as much information as possible which can be used during the matching phase. The matches are created by the teams in Berlin and Romania, but to make their job easier, the technical team in marta compute a "matching score" by which possible matches are sorted. As a result, a number of caregiver profiles with high matching scores will be shown to the family. To quicken the search, filter functionality is also provided. This includes caregiver\'s earliest starting date, German skills, experience with diseases, etc.
