\newpage
\chapter{Introduction}

\section{Background}
It is a common belief in modern society that the more choices, the better--that the human ability to manage, and the human desire for, choice is unlimited. One study showed that the existence of choice increases motivation and enhances performance on doing tasks \autocite{zuckerman1978importance}. However, another study has shown that although having more choices might appear desirable, it may sometimes have negative effects on human motivation \autocite{iyengar2000choice}. In our digital and website-driven era, these studies can be applied on e-commerce and one of the solutions to this problem is a search-and-filter functionality. Amazon is one of the corporate giants for e-commerce, that implemented this feature in their platform. Another example would be marta.
Marta as a business is currently best described as a marketplace between caregivers and families requiring 24-hour care. 24-hour care can be defined as living in a household with the person in need of care for a certain period of time. This means that caregivers are primarily responsible for basic care and household chores. In addition, they support the person in care's relatives in need of assistance in carrying out the activities they wish to do. Marta as a marketplace connecting families with caregivers is competing against more traditional agencies, where it can take several days or even weeks to find a family for a newly signed up caregiver or the other way around.

\section{Problem Statement and Objectives}
In this rapidly growing world, human has access to almost everything. On the other hand, having too many options may cause negative effects on the human motivation. Imagine browsing for a cheap, small vacuum cleaner on Amazon to replace your trusty 5-years-old broom. You'd find a thousand cheap small vacuum cleaners from different brands, even in different colors. You'd end up spending most of your time comparing which vacuum cleaner would deliver the best performance, of any sort, for its price, even though all of those vacuum cleaners are exactly what you wanted, cheap and small. As a multinational technology company, Amazon aims to allow users to complete a transaction as quickly as possible. In order to achieve this goal, Amazon introduced a search-and-filter bar in their website. It sounds like a fair idea at first, but if users searched for a similar item over and over again, they would need to type in the same letters or words over and over again.
Similar to Amazon, marta would need to enhance their product, to compete in this expeditiously developing business. One way marta can provide a superior experience for both caregivers and families is to speed up the matching process. The caregiver and family inquiry forms are designed to record as much information as possible which can be used during the matching phase. The matches are created by the teams in Berlin and Romania, but to make their job easier, the technical team in marta compute a "matching score" by which possible matches are sorted. As a result, a number of caregiver profiles with high matching scores will be shown to the family. To quicken the search, filter functionality is also provided. This includes caregiver's earliest starting date, German skills, experience with diseases, etc.
Marta needs to introduce a continually improved, user-friendly filtering functionality to adjust to the needs of families. An example of a user-friendly filter would be to provide quick filter suggestions which the user can click once and the desired results will be shown, instead of letting users select the same filter manually over and over again. In order to determine which filter suggestions are the most beneficial, frequently used filters need to be identified.

\section{Structure of the Thesis}
The bachelor thesis is structured as follows: Besides the introduction in chapter 1, browser extensions are discussed in more detail in Chapter 2 along with the reasons why they are a suitable solution to this issue. Additionally, it describes the extension's architecture and the technologies that are paired with to create the application. Based on this, Chapter 3 describes the experimental methodology used for studies conducted with the Chrome extension. The extension's implementation and use in a practical situation are discussed in some detail in Chapter 4 along with some lessons learned. Subsequently, the extension is evaluated and analyzed. Finally, problems are identified and an outlook for possible improvements of the extension is given.
