\newpage
\section{Introduction}

\subsection{Background}

Marta as a business is currently best described as a marketplace between caregivers and families requiring 24-hour care. 24-hour care can be defined as living in a household with the person in need of care for a certain period of time. This means that caregivers are primarily responsible for basic care and household chores. In addition, they support the person in care\'s relatives in need of assistance in carrying out the activities they wish to do. Marta as a marketplace connecting families with caregivers is competing against more traditional agencies, where it can take several days or even weeks to find a family for a newly signed up caregiver or the other way around.

One way marta can provide a superior experience for both caregivers and families is to speed up the matching process. The caregiver and family inquiry forms are designed to record as much information as possible which can be used during the matching phase. The matches are created by the teams in Berlin and Romania, but to make their job easier, the technical team in marta compute a "matching score" by which possible matches are sorted. As a result, a number of caregiver profiles with high matching scores will be shown to the family. To quicken the search, filter functionality is also provided. This includes caregiver\'s earliest starting date, German skills, experience with diseases, etc.

\subsection{Problem Statement and Objectives}

It is a common belief in modern society that the more choices, the better--that the human ability to manage, and the human desire for, choice is unlimited. One study showed that the existence of choice increases motivation and enhances performance on doing tasks \cite{zuckerman1978importance}. However, another study has shown that although having more choices might appear desirable, it may sometimes have negative effects on human motivation \cite{iyengar2000choice}. In our digital and website-driven era, these studies can be applied on e-commerce and a solution to this problem is a search-and-filter functionality. Amazon is one of the corporate giants for e-commerce, that implented this feature in their platform.

In our case, marta needs to introduce a continually improved, user-friendly filtering functionality to adjust to the needs of families. An example of a user-friendly filter would be to provide quick filter suggestions which the user can click once and the desired results will be shown, instead of letting users select the same filter manually over and over again. In order to determine which filter suggestions are the most beneficial, frequently used filters need to be identified.

\subsection{Thesis Structure}% TODO: change subsection name
% TODO: thesis structure
The bachelor thesis is structured as follows: Besides the introduction in chapter 1, chapter 2 explains the theoretical basics and describes the hardware and software used. Based on this, Chapter 3 describes the design and then, in Chapter 4, the implementation. Subsequently, the system is evaluated and analyzed. Finally, problems are identified and an outlook for possible extensions of the system is given.
