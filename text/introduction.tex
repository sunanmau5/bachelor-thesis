\newpage
\chapter{Introduction}
Direct access to digital information has completely changed how a single piece of information is retrieved, enabling us to be "better than reality" with search boxes, allowing users to jump directly to what they are interested in without using any complex systems. This instant gratification far outperforms previous methods of flipping through physical pages. However, the spread of digital information access has been accompanied by an explosion in the volume of information available about any given topic, to the point where even instant access via search does not necessarily make finding our needle in a haystack any easier.

It is a common belief in modern society that the more choices, the better--that the human ability to manage, and the human desire for, choice is unlimited. One study showed that the existence of choice increases motivation and enhances performance on doing tasks \autocite{zuckerman1978importance}. However, another study has shown that although having more choices might appear desirable, it may sometimes have negative effects on human motivation \autocite{iyengar2000choice}. To resolve this issue, filters are introduced. Filters help users find informations faster. Rich information systems have recently begun to provide faceted navigation, which basically extends the idea of filters into a complex structure that attempts to describe all the different aspects of an object in order to maximize flexibility in retrieving information \autocite{whitenton2014filters}. Nevertheless, using this more flexible and more useful tool requires multiple steps, expecially if users repeatedly search for the same information.

\section{Objectives}
The goal of this thesis is to design and implement a filtering suggestions tool, consisting of a client component, to circumvent the numerous steps involved in searching for an article or a product on a website. Thereby, a clear presentation and its development will be discussed as well as the technical background of the extension. The client component is implemented as a Google Chrome extension. To achieve a user-friendly extension, it is critical to provide the user with access to useful and clear information, so the extension to be developed will be compactly customized to one view, allowing the user to see their frequently visited websites.

The original goal of this thesis is to create an additional feature for marta's customer-facing platforms. Marta as a business is currently best described as a marketplace between caregivers and families requiring 24-hour care. 24-hour care can be defined as living in a household with the person in need of care for a certain period of time. This means that caregivers are primarily responsible for basic care and household chores. In addition, they support the person in care's relatives in need of assistance in carrying out the activities they wish to do. Marta as a marketplace connecting families with caregivers is competing against more traditional agencies, where it can take several days or even weeks to find a family for a newly signed up caregiver or the other way around. As a growing start-up that gained a lot of users in the past few months, marta would need to enhance their product, to compete in this expeditiously developing business. One way marta can provide a superior experience for both caregivers and families is to speed up the matching process. The caregiver and family inquiry forms are designed to record as much information as possible which can be used during the matching phase. The matches are then created by the teams in Berlin and Romania, but to make the job easier, the technical team in marta compute a "matching score" by which possible matches are sorted. As a result, a number of caregiver profiles with high matching scores will be shown to the family. To quicken the search, filter functionality is also provided. This includes caregiver's earliest starting date, German skills, experience with diseases, etc.

Marta needs to introduce a continually improved, user-friendly filtering functionality to adjust to the needs of their users. An example of a user-friendly filter would be to provide quick filter suggestions which the user can click once and the desired results will be shown, instead of letting users select the same filter manually over and over again. In order to determine which filter suggestions are the most beneficial, frequently used filters need to be identified.

\section{Structure of the Thesis}
The thesis is divided into four chapters. First, the descriptions of filtering, exploratory search and faceted search are provided. In addition, the anatomy of a URL as well as search and filtering usage in a URL are defined. Finally, the fundamentals of the technologies used are then explained in chapter 2. Further on, in chapter 3 the experimental methodology used for studies conducted with the Chrome extension is described. The extension's implementation and use in a practical situation are discussed in some detail in chapter 4 along with some lessons learned. Subsequently, the extension is evaluated and analyzed. Finally, problems are identified and an outlook for possible improvements of the extension is given.
