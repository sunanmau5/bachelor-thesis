\newpage
\chapter{Methodology}

\section{Requirements Elicitation and Analysis}
% TODO: what are the requirements?
The software requirements are classified into functional requirements and non-functional requirements.

\subsection{User Requirements}
% TODO: the services that the system should provide and the constraints under which it must operate. We don’t expect to see any level of detail, or what exactly the system will do, It’s more of generic requirements.

\subsection{System Requirements}
% TODO: a more detailed description of the system services and the operational constraints such as how the system will be used and development constraints such as the programming languages.

\subsection{Functional Requirements}
% TODO: main functions that should be provided by the system.
A requirement is called functional if its underlying need is functional, i.e., it relates to information processing objects (data, operations, behavior). In other words, functional requirements are statements of services the system should provide, how the system should react to particular inputs and how the system should behave in particular situations \autocite{sommerville2011software}.
\begin{itemize}
  \item FR-01: User can select which
  \item FR-02: User can
  \item FR-03: User can
  \item FR-04: User can
\end{itemize}

In addition to functional requirements, non-functional requirements are also defined as follows.

\subsection{Non-Functional Requirements}
% TODO: the constraints on the functions provided by the system.
A requirement is called non-functional if its underlying need is a non-objective property. In other words non-functional requirementes are constraints on the services or functions offered by the system such as timing constraints, constraints on the development process, standards, etc \autocite{sommerville2011software}. It often apply to the system as a whole rather than individual features or services.

\begin{itemize}
  \item NFR-01: UI components must be implemented using React Typescript.
  \item NFR-02: JavaScript must be used to implement dynamic functions.
  \item NFR-03: Both the external components used and the developed component itself must be well documented.
  \item NFR-04: The individual external libraries must be easy to update.
  \item NFR-05: Modularization must be taken into consideration throughout development so that individual modules and/or components can be easily reused, expanded or changed.
\end{itemize}

\section{Design Concept}
% TODO: what is the design concept?
